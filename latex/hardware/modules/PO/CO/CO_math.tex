\subsubsection {Configuring thresholds for the non-inverting comparator} \label{threshold}

$U_1$ is a non-inverting comparator with hysteresis. $R_{1}$ and $R_{2}$ together with the
bias voltage \qty{1.9}{\V}
select the lower and upper tripping voltages.
The computation of thresholds for the non-inverting comparator configuration with hysteresis is shown
in several documents produced by TI:

\begin{enumerate}
    \item  equation (4) in datasheet \cite{noauthor_tlv703x_2021}: this equation   seems to be incorrect.
    \item  TI application notes SBOA313A and TIDU020A.
\end{enumerate}

I will use the standard approach to circuit
analysis with Kirchhoff Voltage Law (KVL):

We want the hysteresis function shown in Fig \ref{fig:ch} with lower voltage threshold $V_L$ and higher voltage threshold $V_H$:


\begin{figure}[h]
    \centering
    \includegraphics[width=.4\textwidth]{hysteresis}
    \caption{Comparator hysteresis}
    \label{fig:ch}
\end{figure}


We search for $V_H$, the voltage where the comparator output will transition from low to high.

$V_{TH}$ is the threshold voltage applied to the inverting input of the comparator. This is also known bias voltage.




\begin{figure}[h]
    \centering
    \begin{subfigure}[b]{0.4\textwidth}
        \centering
        \includegraphics[width=\linewidth]{circuits/hysteresis1.pdf}
        \subcaption{KK for low-high transition}
        \label{fig:lh}
    \end{subfigure}
    \hfill
    \begin{subfigure}[b]{0.45\textwidth}
        \centering
        \includegraphics[width=\linewidth]{circuits/hysteresis2.pdf}
        \subcaption{KK for high-low transition}
        \label{fig:hl}
    \end{subfigure}
    \caption{Equivalent circuits for hysteresis equations}
    \label{fig:ui}
\end{figure}


Applying KKL for Fig \ref{fig:lh} gives us:



\begin{align*} \label{lh}
    V_{TH} - V_H + R_1 I_1 = 0                   \tag*{mesh {$M_I$}}    \\
    V_{TH} + R_2 I_2 = 0                         \tag*{mesh {$M_{II}$}} \\
    I_1 + I_2 = 0                                                       \\
    \Rightarrow V_{TH} = V_H \frac{R_2}{R_1 + R_2}                      \\
    \text{let $k$} = \frac{R_2}{R_1 + R_2}                              \\
    \Rightarrow V_{TH} = V_H k                           \eqnumtag\label{eqn:lh}
\end{align*}




Applying KKL for Fig \ref{fig:hl} yields:


\begin{align*} \label{hl}
    V_{TH} - V_L + R_1 I_1 = 0               \tag*{mesh {$M_I$}}                       \\
    V_{TH} - V_{cc}  + R_2 I_2 = 0                              \tag*{mesh {$M_{II}$}} \\
    I_1 + I_2 = 0                                                                      \\
    \Rightarrow  -V_L + R_1 I_1 + V_{cc} - R_2 I_2     = 0                             \\
    \Rightarrow -V_L + V_{cc} -  I_2 (R_1 + R_2)       = 0                             \\
    \Rightarrow I_2 =  \frac{V_{cc} - V_L}{R_1 + R_2}                                  \\
    \text{let $k$} = \frac{R_2}{R_1 + R_2}                                             \\
    \Rightarrow V_{TH} =  V_{cc}(1 - k) + V_L k              \eqnumtag\label{eqn:hl}
\end{align*}


With eq. \ref{eqn:lh} and \ref{eqn:hl} we must now select values for $R_1$, $R_2$ and $V_{TH}$. In practice, these values
are not real numbers but must be chosen from a finite set of available components. In addition, at least three different
circuit configurations for setting $V_{TH}$ are possible:

\begin{itemize}
    \item a basic voltage divider as show in (SBOA313A). The trade-off here is that larger resistors introduce more noise and
          smaller resistors increase power consumption. Since our application is subjected to large temperature variations,
          the resistors should not only feature tight tolerances (at least 0.1 \% ) but also a low temperature coefficient.
          Such resistors are expensive.
    \item a voltage divider followed by a low noise op amp in buffer configuration. This choice increases component count and
          cost.
    \item a voltage reference or voltage regulator with low temperature drift.  This reduces component count ( but not necessarily cost)
          and offers the best accuracy. The trade-off here is that only a small set of reference voltages is available as
          single component which means that eq. \ref{eqn:lh} and \ref{eqn:hl} can only be approximated.
\end{itemize}

We chose the last option since it minimizes component count. Selecting voltage regulator $U_7$ with
$V_{TH} = \SI{1.9}{V}$, $R_1 = \SI{129}{\kilo\ohm}$ and $R_2 = \SI{569}{\kilo\ohm}$
approximates the desired tripping voltages
$V_l=\qty{3.7}{\V}$ and $V_h=\qty{3.9}{\V}$ with acceptable accuracy.

This is verified with the help of a circuit simulator.



\subsubsection{Simulation the power supervisor circuit}


We use \href[]{https://www.analog.com/en/design-center/design-tools-and-calculators/ltspice-simulator.html}{LTSpice 17.1.10}.
We simulate $V_L=\qty{2.89}{\V}$ and $V_H=\qty{3.84}{\V}$. Recall that the goal was $V_L=\qty{3.0}{\V}$ and $V_H=\qty{3.9}{\V}$.
This deviation is acceptable for our application.


\begin{figure}[h]
    \centering
    \includegraphics[width=\textwidth]{supervisor-sim}
    \caption{Power supervisor Spice simulation}
    \label{fig:sss}
\end{figure}

