\chapter*{Abstract}

Remote monitoring of drinking water in the context of this project means that the amount of drinking water
used by a group of people is constantly measured in terms of liters per minutes.
The objective is to detect abnormal consumption which could be caused by a pipe leak or by some sort of
exceptional usage by one or more subscribers. In both cases, the origins of this excess consumption
must be analyzed as soon as possible. The monitoring system must immediately inform the person in charge of the water
distribution system, preferably by SMS and/or email. In addition, the monitoring system should provide the data
to build usage profiles. These profiles would show the consumption of drinking water over the course of a day,
during summer and winter and during holidays. The latter is important because the village involved
in this project has considerable touristic activity.

Challenges arise from the lack of energy and of reliable network infrastructure at the point of
metering. The energy powering the
system must  thus be produced off-grid. We have chosen small solar panels.
Data communications rely on GSM, whose coverage
turned out to be intermittent in the location where the system operates.


