\chapter{Software}

The software consists of several components:
\begin{enumerate}

    \item firmware for \mu M and \mu S.
    \item server (Linux daemon) for data retrieval and storage.
    \item command line tools for post-processing and reporting.
\end{enumerate}


\section{Firmware}

The firmware consists of two separate programs that are loaded (flashed) on the master and the slave microcontroller.
Both programs share a set of libraries that provide common functions like logging and user I/O.
Both subsystems write diagnostic data to a SD-card that can be used for debugging.


\subsection{Master firmware}

The master has the following tasks:

\begin{enumerate}
    \item after startup, request the current internet time from the slave
    \item then start incrementing the flow counter based on the voltage readings from the IR opto switch
    \item at a given time - currently at 12pm - transmit the data to the slave for upload to the MQTT broker
    \item while waiting for the upload to succeed, continue incrementing the flow counter
    \item if the slave acknowledges a successful upload, reset the counter and continue
    \item if the slave does not succeed, retry later and continue
\end{enumerate}

Let us further detail  these tasks.




At the end of the startup routine the following conditions hold:
\begin{itemize}
    \item [-] the \hyperref[sec:WD]{Watchdog Module (WM)} is enabled (.MA.WD.EWD = H).
    \item [-] the  \hyperref[sec:SL]{Slave Module ((\mu S))} is powered of (.MA.SS.ESS = L).
\end{itemize}

The master then goes into power-saving sleep mode and wakes up every $n$ seconds to perform the following actions:
\begin{itemize}
    \item [-] toggle MA.WD.\neg RS to prevent a watchdog timeout and a subsequent reset.
    \item [-] read the analog opacity level from module OS with the following sequence: \\
          .MA.OD.EOD = L, read analog input MA.OD.Prx, .MA.OD.EOD = H.
    \item [-] read the digital input MA.BI.CON' to check if the slave is ready to receive commands via the I2C bus.
\end{itemize}





\subsubsection{Internet Time}

Accurate time is required for logging (timestamp) and to initiate the transmission request at a specific time.
The time is updated only at startup, there will be some drift over time based on the stability of the microcontroller RTC.
This is acceptable for this application.

\subsubsection{Robust sensor reading}
\label{sec:rsr}

The voltage read from the IR opto switch oscillates between a maximum and minimum voltage.
To increase robustness with respect to noise, a lower and higher threshold are chosen. Further, a hysteresis function
must be applied to avoid erroneous transitions in between the maximum/minimum values.
The question is now how to choose the lower and upper thresholds as well as the hysteresis.
As explained in chapter \ref{sec:osd} the sensor readings are subject to drift and it is therefore desirable to
increase the current through the sensor to decrease the impact of drift by widening the gap between lower and
higher threshold. This however, increases power consumption. From the perspective of power consumption,
it would therefore be optimal to drive just enough current through the sensor to allow for unambiguous discrimination
of upper and lower thresholds.

To mitigate the effect of drift, we use an averaging algorithm that constantly adjusts the upper and lower thresholds:


\par\noindent\rule{\textwidth}{0.4pt}

\begin{figure}[h]
    \centering


    \begin{algorithmic}
        \State $current_{min} \gets high_{rail}$
        \Comment{upper rail, Vcc, supply voltage}
        \State $current{max} \gets low_{rail}$
        \Comment{lower rail, Gnd, 0 V}
        \State $\text{lower threshold breached} \gets false$
        \Comment{have we already seen the lower threshold ?}
        \Loop
        \Comment{every second}
        \State $current \gets A_0$
        \State $current_{min} \gets min(current, current_{min})$
        \Comment{search for new min}
        \State $current_{max} \gets max(current, current_{max})$
        \Comment{search for new max}
        \If{$n\pmod{N} = 0$}
        \Comment {every Nth invocation}
        \State update $low$
        \Comment {apply hysteresis}
        \State update $high$
        \Comment {apply hysteresis}
        \State $current{min} \gets high_{rail}$
        \Comment {reset}
        \State $current{max} \gets low_{rail}$
        \Comment {reset}
        \EndIf
        \If {$(current < low)$ $(\wedge$ $\neg$ \text{lower threshold breached})}
        \State inc
        \BoxedString[fill=yellow]{counter}
        \State $\text{lower threshold breached} \gets true$
        \ElsIf {$current > high$}
        \State $\text{lower threshold breached} \gets false$
        \EndIf
        \EndLoop

    \end{algorithmic}
    \caption{Adaptive threshold algorithm}
\end{figure}

\subsubsection{Power sensitive operation}

Upload only if battery voltage is high enough.

\subsubsection{Master Slave communication}

The I2C bus is used for bidirectional communication with the help of the Arduino \texttt{Wire} library:

\begin{enumerate}
    \item the master, during startup initializes the bus with a call to \texttt{Wire.begin()}.
    \item the slave, during startup joins the bus with a free I2C address \texttt{Wire.begin(address)}.
    \item the slave installs event handlers to receive data and to receive requests to send data.
    \item the slave interconnects \texttt{SDA} and \texttt{SCL} between master and slave by pulling up $D_0$.
    \item the master - and only the master - initiates the communication by either sending data with \texttt{Wire.write(data)}
          or requesting data with \texttt{Wire.requestFrom(address,nb of bytes)}.
\end{enumerate}



\subsection{Slave firmware}

At specific points in time, the master transmits the meter reading to the slave for upload to a MQTT broker.
Currently, the time interval is fixed (every 24 hours) but a more sophisticated strategy would be to make
this interval dependent on the current charge level of the battery. That is in summer - where leaks are more disturbing and
energy is plentiful - more frequent transmissions could be scheduled than in winter where the conditions and
requirements are
very different.

In our remote environment with relatively pour radio coverage we have found it challenging to upload data via
General packet radio service (GPRS).
Firstly, connection attempts frequently fail and no data is transmitted at all.
Secondly, the Arduino \texttt{MQTT} library might return one of the following error codes despite the fact
that the upload has been successful:

(This was with Qos = 1 )

\begin{minted}{c}
LWMQTT_NETWORK_FAILED_CONNECT = -3,
LWMQTT_NETWORK_TIMEOUT = -4,
LWMQTT_NETWORK_FAILED_READ = -5,
LWMQTT_MISSING_OR_WRONG_PACKET = -9,
\end{minted}


And lastly, based on the return value provided by \texttt{MQTT},
the slave might think that the transmission was successful when in fact it was not.

\subsection{Choosing MQTT parameters}

For our purposes, the following parameters are important:
\begin{enumerate}
    \item Qos (Quality of Service): we use \texttt{Qos = 0} because as explained before the
          handshake operation does  not seem to work reliably, given the intermittence of the radio link.
    \item Retained messages: we use \texttt{retained = true} to garantee that the server can connect and
          subscribe without blocking
    \item Persistent Session/Queued Messages: we use \texttt{cleanSession = true}
          because we do not need to store subscription information or any other information across multiple TCP connections.
\end{enumerate}
\ \\

The most important MQTT feature for our application is Retained Messages.
This feature guarantees that when the server connects, it will always see the last data uploaded by the slave.
Retained Messages
provide a start state.
Note the \href{https://www.youtube.com/watch?v=Ct5s4gXefn4&list=PLRkdoPznE1EMXLW6XoYLGd4uUaB6wB0wd&index=9}{difference}
between Retained Messages and Queued Messages.

Retained Messages
\begin{itemize}
    \item work on a topic level
    \item newly-connected subscriber to a topic receive a message immediately
\end{itemize}

Queued Messages
\begin{itemize}
    \item work in a client context
    \item the broker queues undelivered messages for a specific client
\end{itemize}

We do not need to queue messages, because we know exactly how many messages the slave will send in a given timeframe.
Therefore, the sample interval of the server can be chosen such that messages are never lost.


To make the upload procedure reliable we chose the following approach:


\par\noindent\rule{\textwidth}{0.4pt}

\begin{figure}[h]
    \centering
    \begin{algorithmic}
        \State $success = false$
        \ForAll{$n \in \{1, \dots, 6\} \wedge success$}
        \State open GPRS connection
        \State open subscribe to topic \texttt{water\_meter}
        \Comment{subscribe before publish !}
        \State publish meter data to topic \texttt{water\_meter}
        \State $received \gets echo$ from subscription
        \Comment{this only happens when the upload was successful}
        \State $success = (sent == received)$
        \Comment{we are finished when the echo matches}
        \EndFor
        \If{success}
        \State $master \gets success$
        \Else
        \State $master \gets failure$
        \EndIf
    \end{algorithmic}

    \caption{Slave Broker handshake operation}
\end{figure}

Once the master receives either \texttt{success} or \texttt{failure}, the slave is powered off.

A slight disadvantage of this handshake operation is that the message send from the broker to the slave
gets lost. In this case, the slave will attempt another (paid) transmission despite the fact that the previous upload
has been successful. Our strategy is therefore conservative which matches the requirements.


\section{Server software}

This component is available as a Linux daemon managed with the \texttt{systemctl} facility which is standard on
current Linux desktop and server stations. The software is meant to be installed on an always-on box and should be
configured via \texttt{systemctl} to start automatically on boot. The software is written in Java.

The daemon has the following tasks:

\begin{enumerate}
    \item poll the MQTT broker for a new upload from the slave
    \item compensate the reading for possible gaps in the upload sequence
    \item store the data in a local database
    \item upload the data to a cloud-based database such that data can be visualized in a static webpage
    \item alert the supervisor in case of abnormal readings (leak)
\end{enumerate}


\subsection{Getting data from the broker}


When using a publicly accessible broker, one must bear in mind that this doesn't come for free. While there
are free solutions, we found them unsuitable for a production scenario. Since MQTT is a publish-subscribe framework,
the obvious way to get data updates is to subscribe to the relevant topic. However, the broker has to do work
to maintain the subscription connection and this translates in transactions that are billed.



We therefore do not entertain a permanent subscription with the broker in the same way
as we would choose if that broker were deployed locally where we would not incur communication costs.
Rather, we connect and subscribe only at specific time points.
Since the slave sends the message with the retained flag set, the server is guaranteed to get an update
instantaneously (after at least one upload has been performed by the slave). In other words, the \texttt{Runnable} passed
to the \texttt{ScheduledService} will never block. This approach resembles more to polling than to publish/subscribe.
As long as the polling interval is shorter than the upload interval, we are guaranteed not to lose any data.
Here, we connect every 12 hours. Recall that the slave uploads every 24 hours.

Here is an extract from the server code:

\begin{figure}[h]
    \centering
    \begin{minted}
        [frame=lines,
framesep=2mm,
baselinestretch=1.2,linenos
            ]
        {java}
Runnable r = () -> {
			var cloud = io.smq().get();
			Effect<Data_Arduino> e_inc = data -> {
				cloud.disconnect();
				pf.put(f_yi_sensor(io::y).apply(data));
			};
			cloud.sub(arduino_mqtt_topic, fd_broker, e_inc); 
		};
        ScheduledFuture<?> update_data = io.fses().apply(r);
        \end{minted}
    \caption{Polling-style publish/subscribe operation}
\end{figure}

\FloatBarrier
\paragraph*{Comments}:
\begin{enumerate}
    \item [1:] create a \texttt{Runnable}. The code in curly brackets will not execute until passed to the scheduler in line 9.
    \item [2:] open the connection with the MQTT broker. A \texttt{Supplier} is used so that we can pass in a test mock-server.
    \item [3-6:] event handler that will run when the server receives a message on topic \texttt{arduino\_mqtt\_topic}. Note
          that we disconnect from the broker as soon as we receive the message.
    \item [7:] subscribe to the broker and receive a message immediately (without blocking) because of the Retained Message
          mechanism explained earlier.
    \item [9:] submit the \texttt{Runnable} created in 1. to the scheduler service (again obtained in a way that simplifies tests).
          The code will run at predefined intervals.

\end{enumerate}


