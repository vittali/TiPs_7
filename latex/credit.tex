\chapter*{Credits}


L’équipe PGSSE des communes de Le Puid, Le Vermont et de Saint-Stael is a small team of volunteers in northeastern France.
PGSSE (\textit{Plan de gestion de la sécurité sanitaire des eaux}) is a policy signed in law by the French government
to ensure the quality and availability of drinking water on the entire french territory. The PGSSE is based on
well-known international standards like IOS9001. All communities with more than 50 inhabitants must be certified according
to the rules defined in this standard.

The team is currently comprised of the following individuals:

Francois Boulet, Gilles Droin, Jacotte Kinsk,  Patrick Lorin, Jean-Gilles Naivin, Patrice Omurbek, Peter Vittali.



Gilles Droin has laid the groundwork for this project by founding the
association \textit{L'Atelier d'Acccompagnement Numérique}. This organization helps
small, rural, communities to become more independent with respect to theirs purchase and usage of
IT equipment. In particular, the transition to the Linux operating system is encouraged. It is an initiative
to deploy open source technology in rural communities.


Many thanks also to Régine Chinouilh, major of Le Puid / Vosges, who tolerated many failed
experiments and who was willing to deal with complaints from locals who didn't understand
why we were degrading their beautiful environment with solar panels, plastic conduits and cables.
