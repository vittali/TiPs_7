\chapter{Introduction}


A rural village in north-eastern France is supplying roughly 100 inhabitants with drinking water from local sources.
These sources have been built throughout the 70ies and have so far never run dry.
In recent years, however, signs of reduced flow have been observed and a couple of actions have been taken
to prevent outright interruptions of the water supply to the village.

\section{Motivation}
Firstly, leaks in the water distribution network have been repaired. Secondly, a system was designed to
detect new leaks right as they occur, rather than at some point later when supply issues become apparent.
This has several important advantages. It is sometimes easier to locate the exact position of the leak when
the advent of the leak can be time correlated with recent events in the area, for example ongoing construction
work may have damaged a conduit. The water quality might suffer because untreated groundwater
can infiltrate the water distribution network through the defect conduit.
And lastly, initiating repairs only when absolutely necessary means that these repairs are then
started during summer or at the end of the summer when  upstream sources run low and consumption is at it
highest.
This period of time is not ideal for repair work because users are more  impacted  and local repair services will be
strained because many other villages in the area will experience exactly the same problems.
This situation is further aggravated by the fact that the current water conduits are vulnerable to
small dislocations of the surrounding earth mass which increases after periods of drought due
to the desiccation of the soil throughout the summer.

Preventive repairs are therefore mandatory to maintain a reliable water distribution infrastructure.

Another aspect of drinking water management is consumption management. Currently, subscribers pay a yearly
lump sum and there are no individual water meter inside each subscriber home. This situation is certain
to change, and it might be interesting to understand how this change influences consumption patterns.
Consumption patterns are also important for conduit dimensioning. The required peek throughput is an
important parameter when dimensioning the diameter of water pipes and the size of valves.

\section{Challenges in a rural setting}

I hope to have made clear, why flow monitoring in the drinking water system described above is necessary.
The question is then: why not simply purchasing a smart-meter or even simpler, entering a service
agreement with a resource management company that collects the data and makes it available in a cloud
environment.
It turns out that while this kind of facility management is increasingly common in cities,
a concept also referred
to as \textit{Smart City}, it is not commonly used in the rural setting that I am referring to here.
There are several reasons for this:

\begin{itemize}
      \item Products are not geared towards small communities: Big players in the area of smart meters tend to offer products for
            industrial applications which do not cater very well to the requirements in our rural context, in particular with respect to cost.
      \item Lack of extensibility: Commercial products in the sector are not open-source, they
            might be protected by patents and even the collected data might be owned by the service provider, not the
            client. The market is dominated by few players with wide strategical moat and little competition.
            This makes it difficult to extend the system. For example, when metering the water flow it might be tempting to also meter
            water quality, temperature, rainfall and other environmental parameters of merit. I believe
            that the open-source/DIY/maker community has an important role to play.
\end{itemize}


